%===========================================================================
% SIMBOLI SISTEMA
%===========================================================================
% 1: scritta a sinistra
% 2: scritta dentro il box
% 3: scritta a destra
%===========================================================================

\newcommand{\SISTEMA}[3]
{
	\begin{center}
	\begin{picture}(270,40)
		%box
		\put(100,5){\line(1,0){80}}
		\put(180,5){\line(0,1){30}}
		\put(180,35){\line(-1,0){80}}
		\put(100,35){\line(0,-1){30}}
		%contenuto box
		\put(110,17){#2}
		%frecce
		\put(80,20){\vector(1,0){20}}
		\put(180,20){\vector(1,0){20}}
		%testo sinistra e destra
		\put(40,20){#1}
		\put(210,20){#3}
	\end{picture}
	\end{center}
}

%===========================================================================
% SIMBOLI SISTEMA SERIE
%===========================================================================
% 1: ingresso
% 2: scritta dentro il primo box
% 3: segnale intermedio
% 4: scritta dentro il secondo box
% 5: uscita
%===========================================================================

\newcommand{\SISTEMASERIE}[5]
{
	\begin{center}
	\begin{picture}(300,40)
		%box1
		\put(50,5){\line(1,0){70}}
		\put(120,5){\line(0,1){30}}
		\put(120,35){\line(-1,0){70}}
		\put(50,35){\line(0,-1){30}}
		%box2
		\put(180,5){\line(1,0){70}}
		\put(250,5){\line(0,1){30}}
		\put(250,35){\line(-1,0){70}}
		\put(180,35){\line(0,-1){30}}
		%frecce
		\put(30,20){\vector(1,0){20}}
		\put(120,20){\vector(1,0){60}}
		\put(250,20){\vector(1,0){20}}
		%ingresso
		\put(5,17){#1}
		%testo 1 box
		\put(60,17){#2}
		%segnale intermedio
		\put(140,28){#3}
		%testo 2 box
		\put(190,17){#4}
		%uscita
		\put(280,17){#5}
	\end{picture}
	\end{center}
}

%===========================================================================
% SIMBOLI SISTEMA PARALLELO
%===========================================================================
% 1: ingresso
% 2: scritta dentro il primo box
% 3: segnale intermedio primo box
% 4: scritta dentro il secondo box
% 5: segnale intermedio secondo box
% 6: uscita
%===========================================================================

\newcommand{\SISTEMAPARALLELO}[6]
{
	\begin{center}
	\begin{picture}(260,100)
		%box1
		\put(80,5){\line(1,0){70}}
		\put(150,5){\line(0,1){30}}
		\put(150,35){\line(-1,0){70}}
		\put(80,35){\line(0,-1){30}}
		%box2
		\put(80,45){\line(1,0){70}}
		\put(150,45){\line(0,1){30}}
		\put(150,75){\line(-1,0){70}}
		\put(80,75){\line(0,-1){30}}
		%frecce
			%freccia sopra sinistra
			\put(40,60){\vector(1,0){40}}
			%freccia sotto sinistra
			\put(50,20){\vector(1,0){30}}
			%linea sinistra verso il basso
			\put(50,60){\line(0,-1){40}}
			%freccia uscita box1
			\put(150,60){\vector(1,0){56}}
			%linea uscita box2
			\put(150,20){\line(1,0){60}}
			%freccia destra verso l'alto
			\put(210,20){\vector(0,1){36}}
			%freccia uscita
			\put(214,60){\vector(1,0){20}}
		%simbolo somma
		\put(210,60){\circle{8}}
		%ingresso
		\put(10,57){#1}
		%testo 1 box
		\put(90,57){#2}
		%segnale intermedio 1
		\put(160,70){#3}
		%testo 2 box
		\put(90,17){#4}
		%segnale intermedio 2
		\put(160,30){#5}
		%uscita
		\put(240,57){#6}
	\end{picture}
	\end{center}
}


%===========================================================================
% SIMBOLI NUOVI
%===========================================================================

\newcommand{\sinc}[0]{\mbox{sinc}}
\newcommand{\tri}[0]{\mbox{tri}}
\newcommand{\rect}[0]{\mbox{rect}}
\newcommand{\DTFT}[0]{\mbox{DTFT}}
\newcommand{\DFT}[0]{\mbox{DFT}}
\newcommand{\convcirc}[0]{\otimes}
\newcommand{\trasfz}[0]{\stackrel{\mbox{Z}}{\longleftrightarrow}}
